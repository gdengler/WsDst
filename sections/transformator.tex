\section{Transformator}
		
	\subsection{Trafomodelle}
%		\subsubsection{Grundstrucktur des Trafo}
%			\begin{tabular}{p{7cm}p{4.5cm}p{5cm}}
%	        	\textbf{} &
%	        		\begin{minipage}{4.5cm}
%		            	\includegraphics[width=3.5cm]{bilder/GrundstruckturTrafo.png}
%		            \end{minipage} \\
%	        \end{tabular}
		\subsubsection{Der ideale Trafo}
			\begin{tabular}{p{7cm}p{4.5cm}p{5cm}}
				Übersetzungsverh\"altnis &
					$\text{ü} = \frac{|\underline{U}_1|}{|\underline{U}_2|} =
					\frac{|\underline{I}_2|}{|\underline{I}_1|} = \frac{N_1}{N_2}$ &
					\begin{minipage}{4.5cm}
						\includegraphics[width=3.5cm]{bilder/GrundstruckturTrafo.png}
					\end{minipage} \\
				Leistungsbilanz &
					$\underline{S}_1 = \underline{S}_2$ 
					oder $\underline{U}_2 \cdot \underline{I}_2^* = \underline{U}_1 \cdot \underline{I}_1^*$ \\
				Scheinwiderstandsübersetztung &
					$\underline{Z}_{aU} = \frac{|\underline{U}_1|^2}{|\underline{U}_2|^2} \cdot \underline{Z}_a = \text{ü}^2 \cdot \underline{Z}_a$ &
					\begin{minipage}{4.5cm}
	            		\includegraphics[width=3.5cm]{bilder/IdealerTravoImpedanzwandler.png}
	            	\end{minipage} \\
	            Windungszahl berechnen & $U_{1,eff}= \sqrt{2} \cdot \pi \cdot N_1 \cdot f \cdot A \cdot \hat B$
			\end{tabular}

		
		\subsubsection{Verlustloser und Streuungsfreier Trafo (VST)}
			\input{sections/snipptes/trafo} 
		\renewcommand{\arraystretch}{1.5}	
		\subsubsection{Transformatoren-Hauptgleichung (gilt bei Leerlauf)}
			\begin{tabular}{p{7cm}p{4cm}p{6cm}}
      			$\boxed{U_{10} = \frac{2\pi}{\sqrt{2}}N_1 \cdot f \cdot \hat{B}_1 \cdot A}$
      			& 	$|\hat{u}_{10}| = \hat{i}_0 \cdot \omega \cdot L_1$ & Mit $\omega = 2\pi \cdot f$ und $L_1 = N_1 \cdot \frac{\hat{\Phi}_{10}}{\hat{i}_0}$ folgt: \\
      		
				$\boxed{U_{20} = \frac{2\pi}{\sqrt{2}}N_2 \cdot f \cdot \hat{B}_1 \cdot A}$
				& $|\hat{u}_{10}| = 2\pi N_1 \cdot f \cdot \hat{\Phi}_{10} $
            	& Leerlauferregerfluss durchsetzt magn. Kreis $\bot$:$ \hat{\Phi}_{10} = \hat{B}_1 \cdot A $. So folgt: \\
				
				wobei $\frac{2\pi}{\sqrt{2}} = 4.44$ und $\hat{B} \cdot A = \hat{\Phi}$ &
				$|\hat{u}_{10}| = 2\pi N_1 \cdot f \cdot \hat{B}_1 \cdot A	$
			\end{tabular}
		\renewcommand{\arraystretch}{1}	
		
	\newpage
	\subsection{Der reale (einphasige) Transformator}
		Für dreiphasige Trafos müssen die angegebenen Werte zuerst normiert werden (siehe Kapitel 3).
	
		\renewcommand{\arraystretch}{1.25}
		\subsubsection{Ersetzen der magnetischen Kopplung}
			\begin{tabular}{p{5.8cm}p{7.3cm}p{4.5cm}}
            	Spannung, Strom &
            		$U_2' = U_2 \cdot \frac{N_1}{N_2}$ \quad 
            		$I_2' = I_2 \cdot \frac{N_2}{N_1}$ \\
            	Nennstrom & 
            		$I_N = S_N / U_N$ \\
            	Widerstand, Streufluss &
            		$R_2' = R_2 \cdot (\frac{N_1}{N_2})^2$ \quad 
            		$X_{\sigma 2}' = X_{\sigma 2} \cdot (\frac{N_1}{N_2})^2$ \\
            	Vollst\"andiges Ersatzschaltbild &
            	\begin{minipage}{7cm}
            		\adjustbox{width=7cm}{\input{tikz/trafo_ersatzschaltbild}}
            	\end{minipage}
            	&
					\begin{minipage}{4.5cm}
                    	\tiny
                    		$R_1, R_2'$: Widerstand Spule\\ \\
                    		$jX_{\sigma 1}, jX_{\sigma 2}'$: Streufluss Spule\\ \\
                    		$R_{Fe}$: Eisenverlust\\ \\
                    		$jX_h$: Hauptfluss Spule\\
                    \end{minipage} \\ \\
				Systemgleichung des realen Trafo &
					$\underline{U}_1 = R_1\cdot\underline{I}_1 + jX_{\sigma 1}\cdot\underline{I}_1 + jX_h\cdot(\underline{I}_1+\underline{I}_2')$ \\
					& $\underline{U}_2' = R_2'\cdot\underline{I}_2' + jX_{\sigma 2}'\cdot\underline{I}_2' + jX_h\cdot(\underline{I}_1+\underline{I}_2')$

            \end{tabular}
            
            	Für mittlere Transformatorenleistungen erbgeben sich etwa folgende Relationen zwischen
            	den einzelnen $R$s und $X$s:
            	$$\boxed{ R_1 : R_2' : X_{\sigma 1} : X_{\sigma 2}' : X_h : R_{Fe} \approx 1:1:2:2:1000:10000}	$$
    
        \renewcommand{\arraystretch}{1.5}   
		\subsubsection{Leerlauf und Magnetisierung}
			\begin{tabular}{p{5cm}p{6cm}p{7cm}}
            	Rechnen &
            		\begin{minipage}{13cm}
                    	Mit Leerlaufdaten $R_{Fe}$ und $X_h$ ausrechnen. \underline{$R_1$ und $X_{\sigma1}$ vernachl\"assigen.}
                    \end{minipage} \\ \\
            	Induktiver Leerlaufstrom &
            		$\underline{I}_{10} = \underline{I}_{1Fe} + \underline{I}_{1\mu}$ \quad $(\underline{I}_{1\mu} \gg \underline{I}_{1Fe})$ &
            		\begin{minipage}{7cm}
	            		\adjustbox{width=6cm}{\input{tikz/trafo_leerlauf}}
	            	\end{minipage} \\ \\
				Magnetisierungsstrom &
					\multicolumn{2}{l}{
					$i_\mu = \sqrt{2}I_{\mu 1}\cdot \sin(\omega t) + \sqrt{2}I_{\mu 3}\cdot \sin(3\omega t) + \ldots + \sqrt{2}I_{\mu m}\cdot \sin(m\omega t)$
					\quad$(m=2n+1 \hspace{0.3cm} n\in \mathbb{N}_0)$ 
					}\\
				&
					$I_{\mu RMS} = \sqrt{{I_{\mu 1}}^2 + {I_{\mu 3}}^2 +\ldots+ {I_{\mu m}}^2} $\\ \\
				Leerlaufverluste &
					$P_0 = P_{0Cu} + P_{0Hy} + P_{0Wi}$ \\
				Kupferverluste &
					$P_{Cu} = {I_0}^2 \cdot R_{Cu}$\\
				Hystreseverluste &
					$P_{Hy} \sim f \cdot B^2$ \\
				Wirbelstromverluste &
					$P_W = c_W \cdot f^2 \cdot B^2$ &
					$c_W$ ist materialabh\"angige Konstante \\
				Relativer Leerlaufstrom &
					$i_{0N} = \frac{I_{0N}}{I_{1N}}$ &
					$I_{1N}$ ist eingangsseitiger Nennstrom \\
				Eisenverluststrom &
					$I_{Fe} \approx \frac{P_{0N}}{U_{1N}} \approx I_0 \cdot \cos(\varphi_0)$ \\
				Eisenverlustwiderstand &
					$R_{Fe} \approx \frac{U_{1N}^2}{P_{0N}} \approx \frac{P_{0N}}{I_{Fe}^2}$ \\
				Hauptreaktanz &
					$X_h = L_h \omega = \frac{U_{1N}}{I_{\mu}} = \frac{U_{1N}^2}{Q_{0N}}
					=
					\frac{Q_{0N}}{I_{\mu}^2}$
					& mit $Q_{0N} = \sqrt{S_{0N}^2 - P_{0N}^2}$ \\
				Magnetisierungsstrom &
					$I_\mu = I_{0} \cdot \sin(\varphi_0) = \sqrt{I_0^2 - I_{Fe}^2}$&
					oder $\hat{I}_{\mu} = \frac{\hat{H}_{Fe} \cdot l_{Fe}}{N_1} \Leftrightarrow I_{\mu} = \frac{\hat{H}_{Fe} \cdot l_{Fe}}{\sqrt{2}\cdot N_1}$ \\

				Leistungsfaktor i m Leerlauf &
					$\cos(\varphi_0) = \frac{P_{0N}}{I_{0N} \cdot U_{1N}}$ & \multirow{2}{*}{\adjustbox{height=2cm}{\input{tikz/trafo_zeigerdiagrammRealLeerlauf} }}\\\\
				Leerlaufstrom &
					$I_0 = \sqrt{{I_{\mu}}^2 + {I_{Fe}}^2}$
            \end{tabular}
            
        \renewcommand{\arraystretch}{1.25}
		\subsubsection{Kurzschluss}
			\begin{tabular}{p{5cm}p{6cm}p{7cm}}
				\multicolumn{3}{p{18cm}} {
					Mit Kurzschlussdaten $R_1$ und $X_{\sigma1}$ ausrechnen. \underline{$R_{Fe}$ und $X_h$ vernachl\"assigen.}
				}
			
	            \\
				Kurzschlussimpedanz &
					$\underline{Z}_k = R_k + jX_k = \frac{\underline{U}_k}{\underline{I}_k}$& \multirow{2}{*}{
										\adjustbox{width=6.5cm}{\input{tikz/trafo_kurzschluss}}
											              }\\
					& $Z_k = \frac{U_k}{I_k}$ & \\
					& $R_k = R_1 + R_2' = \cos{\varphi_k} \cdot Z_k  = \frac{P_k}{I_k^2}$  \\
					& $X_k = X_{\sigma1} + X_{\sigma2}' = \sin{\varphi_k} \cdot Z_k = \frac{Q_k}{I_k^2}$ \\
				Leistungsfaktor im Kurzschluss &
					$\cos(\varphi_k) = \frac{P_k}{U_k \cdot I_k} = \frac{P_k}{S_k}$	
					& $U_R = U_k \cos\varphi_k$\\
					& $P_k = I_k^2 \cdot R_k$ 					
					& $U_\sigma = U_k \sin\varphi_k$ \\
					& & $R_k=\frac{U_R}{I_k} \qquad X_k = \frac{U_\sigma}{I_k}$\\
				Dauer-Kurschluss-Strom&
					$ \underline{I}_{1k} = \frac{\underline{U}_1}{\underline{Z}_k}$\\
				Nennkurzschlussspannung&
					$ U_Z = I_{1N} \cdot Z_k$\\	
				&	$ u_z = \frac{U_Z}{U_{1N}}$
             \end{tabular}
		\renewcommand{\arraystretch}{1.5}
	
		\subsubsection{Spannungs\"anderung bei Belastung}
		\textbf{Komplexe Rechnung}\\
			\begin{tabular}{p{5cm}p{5cm}p{6cm}}
	            		$\boxed{\underline{U}_1 =
	            		\underline{U}_R+\underline{U}_X+\underline{U}_2'}$ & 
	            		$\boxed{\underline{I}_2' = \underline{I}_1}$ &
	            		\multirow{3}{*}{
	            		\parbox{6cm}{\includegraphics[width=5cm]{bilder/ErsatzschaltbildTrafoLast.png}}}
	            			            	\\		
	            		$\underline{U}_R=R_k \cdot \underline{I}_1$ &
	            		$\underline{U}_X=jX_k \cdot \underline{I}_1$ \\
	            		$\underline{U}_2'=\underline{U}_2 \cdot "u = \underline{I}_2' \cdot \underline{Z}'  $\\
	            		$\underline{Z}' = \underline{Z} \cdot {"u}^2$ \\
            \end{tabular}\\
		\renewcommand{\arraystretch}{1.0}

		\textbf{Graphische Rechnung}\\
			\begin{tabular}{p{8cm}|p{10cm}}
	 			\textbf{Zeigerdiagramme} & \textbf{Kappsches Dreieck}\\
	 			
				\begin{minipage}{8cm}
	            	\begin{multicols}{3}
	            	  \adjustbox{width=2.4cm, set vsize={7cm}{0cm}}{\input{tikz/zeigerdiagramme/LastOhmsch}}
	            	  
	            	  \columnbreak
	            	  
	            	  \adjustbox{width=2.2cm, set vsize={7cm}{0cm}}{\input{tikz/zeigerdiagramme/LastInduktiv}}
	            	  
	            	  \columnbreak
	            	  
	            	  \adjustbox{width=2.8cm, set vsize={7cm}{0cm}}{\input{tikz/zeigerdiagramme/LastKapazitiv}}\\  	  
	            	\end{multicols}
	            \end{minipage}  & 
	            \hspace{0.2cm} 
	            \vspace{-4cm}
				\begin{minipage}{10cm} 
		        	\begin{minipage}{2.5cm}
						\includegraphics[width=3cm]{bilder/KappschesDreieck.png}
		            \end{minipage}
					\begin{minipage}{7.5cm}
		      			$$u_{\varphi} = u_{\varphi'} + 1 - \sqrt{1 - u_{\varphi''}^2}$$
		      			$$ \boxed{u_\varphi \approx u_{\varphi'}}\quad (\text{für }u_z
		      			=\frac{U_Z}{U_{N}} \cdot 100\% < 4 \%)$$
		      			$$u_{\varphi'} = n \cdot u_r \cdot \cos \varphi + n \cdot u_x
		      			\cdot \sin \varphi$$ $$u_{\varphi''} = n \cdot u_x \cdot \cos \varphi - n \cdot u_r \cdot \sin \varphi$$
		      			$$\text{Lastfaktor } n = \frac{I_1}{I_{1N}}$$
		      			$$ u_r = \frac{R_k \cdot I}{U_N} \qquad u_x = \frac{X_k \cdot I}{U_N} $$
		      			$$ U_2 = \frac{U_1}{"u} - \Delta U_2 = \frac{U_1}{"u} - \frac{1}{"u}
		      			U_N u_\varphi $$
		      			$$ \Delta U_2 \approx \frac{1}{"u} \cdot (U_R \cdot \cos \varphi + U_X \cdot \sin \varphi)$$\\
		      		\end{minipage}         
                \end{minipage}\\
				\begin{minipage}{8cm}
					\vspace*{-2cm}
 					\begin{tabular}[c]{p{2.66cm}p{2.66cm}p{2.66cm}}
                     	$\quad \varphi = 0$ & $\quad\varphi > 0$ & $\quad\varphi
                     	< 0$\\ rein ohmsche & induktive & kapazitive\\
                     	Last & Last & Last\\
                     	&&\\
                     	Konstruktion: & 1. $\underline{U}_2'$ senkrecht&\\
                     	& 2. $\underline{I}_1$ mit $\varphi$ zu $\underline{U}_2'$\\
                     	& \multicolumn{2}{l}{3. $\underline{U}_R$ in Phase mit $\underline{I}_1$ }\\
                     	& \multicolumn{2}{l}{4. Kappsches Dreieck $\Rightarrow \underline{U}_1$}
                     \end{tabular}               
                \end{minipage}& \hspace{0.2cm}
				\begin{minipage}{10cm}
		      		Beim kappschen Dreieck wird mit ``genormten'' Gr\"ossen (klein $u$) gerechnet: 
		      		$u = \frac{U_{Strang}}{U_{N, Strang}} = \frac{U_X}{U_N} \qquad
		      		\boldsymbol{U_N = U_1} $\\ \\ Das Kappsche Dreieck dreht sich um die
		      		Spitze der Prim\"arspannung. Bei konstantem Strom und variablem $\cos(\varphi)$ beschreibt ${u}_2'$
		      		einen
		      		Kreis um die Prim\"arspannung.\\ Bei \textbf{kapazitiver Last steigt}
		      		die Sekund\"arspannung über den Leerlaufwert an. \\ 
		      		\textbf{Ähnlichkeitssatz:}\\
		      		\"Andert sich $U_2$ oder $U_1$ so ändern sich alle Gr\"ossen masst\"ablich\\
		      		Bsp.: $\boxed{U_{2_{neu}} = U_{2_{alt}} \cdot \frac{U_{1_{neu}}}{U_{1_{alt}}}}$ \\              
                \end{minipage}     
            \end{tabular}

		
		\subsubsection{Wirkungsgrad des Trafos}
			\begin{tabular}{p{4.5cm}p{7cm}p{7.5cm}}
            	Wirkungsgrad &
            		$\eta = 1- \dfrac{P_V}{P_B} = 1-\dfrac{P_{V0} + P_{VK} \cdot
            		(\frac{P_B}{P_N})^2}{P_B} $ &
            	\begin{minipage}{7cm}
                	$P_B$ = Betriebsnennleistung = $P_1$ = $S_1 \cdot \cos \varphi$\\
                	$P_{N}$ = Nennleistung\\
                	$P_{V0}$ = Leerlaufverlustleistung = $P_{{Fe}_N}$\\
                	$P_{VK}$ = Kurzschlussverlustleistung = $P_{{Cu}_N}$             	
                \end{minipage}\\ \\
            	 &
            		$\eta = 1 - \dfrac{a + (\frac{P_B}{P_N})^2}{P_B} P_{VK}$ &
            	\begin{minipage}{7cm}
 					$a = \dfrac{P_{V0}}{P_{VK}}$ = Verlustverh"altnis                	
                \end{minipage}\\ \\            		
            	 &
            		$\eta = 1 - \frac{P_{V0}+P_{VK}}{P_B} $ 
            		& (bei Wirkleistungsvollast) \\ \\
            	Maximaler Wirkungsgrad
            	& $S_{\eta-max} = \sqrt{a} \cdot S_N$ \quad $P_{\eta-max} = \sqrt{a}
            	\cdot P_N$
            	& (Kupferverluste = Eisenverluste)
            \end{tabular}	

	\subsection{Drehstrom-Leistungstransformatoren} 
	Angegebene Leistung bezieht sich immer auf alle drei Wicklungen. $\Rightarrow P_{Wicklung} =
	\frac{P_N}{3}$ \\
	Angegebene Spannungen und Str\"ome gelten immer für Aussenleiter. Somit muss immer entweder - je
 nachdem ob Dreieck- oder Sternschaltung vorliegt -	Strom oder Spannung mit Faktor
 $\frac{1}{\sqrt{3}}$ multipliziert werden.
 	\subsubsection{Bauformen}
	 $$\text{Kennzeichnung} = [a][b][c][d] = \begin{cases}
                  [a] = \text{Oberspannungswicklung, Grossbuchstabe (Y,D,III,Z)
                  }\\
                  [b] = \text{Unterspannungswicklung, Kleinbuchstabe (y,d,iii,z) } \\
                  [c] \cdot 30^\circ = \text{Phasenverschiebung zwischen Unter- und Oberspannung }
                  \\ [d] = 0 \text{, falls Neutralleiter herausgeführt (optional)}
                  \end{cases}$$
		\begin{center}
	    	\includegraphics[height=6cm]{bilder/Drehstromtrafo.png}
	    \end{center} 
	    
%TODO: Im Parallelbetrieb

	    
